\section{项目版本号命名规范}
\subsection{版本类型介绍}
\begin{center}
	\begin{tabular}{|l|c|p{10cm}|}  
		\hline
		版本类型 & 版本名称 & 版本介绍 \\ \hline
		测试版本 & & \\ \hline
		& Alphal & 内部测试版 \\ \hline
		& Beta & 外部测试版 \\ \hline
		& M 版 & Milestone,每个开发阶段的终结点的里程碑版本 \\ \hline
		& Trail & 试用版(含有某些限制,如时间、功能,注册后也有可能变为正式版) \\ \hline
		& RC版 & Release Candidate,意思是发布倒计时,该版本已经完成全部功能并清除大部分的BUG。到了这个阶段只会除BUG,不会对软件做任何大的更改 \\ \hline
		& RTM版 & Release To Manufactur,意思是发布到生产商,这基本就是最终的版本 \\ \hline
		& GA版 & Generally Available,最终版 \\ \hline
		正式版本 & & \\ \hline
		& Enhance & 增强版或者加强版 \\ \hline
		& Full version & 完全版 \\ \hline
		& Release & 发行版,有时间限制 \\ \hline
		& Upgrade & 升级版 \\ \hline
		& Retail & 零售版 \\ \hline
		& Plus & 增强版,不过这种大部分是在程序界面及多媒体功能上增强 \\ \hline
	\end{tabular}
\end{center}

\subsection{版本号管理策略}
\begin{enumerate}
\item 项目初版本时,版本号可以为 0.1 或 0.1.0, 也可以为 1.0 或 1.0.0。
\item 当项目在进行了局部修改或 bug 修正时,主版本号和子版本号都不变,修正版本号加1。
\item 当项目在原有的基础上增加了部分功能时,主版本号不变,子版本号加1,修正版本号复位为0,因而可以被忽略掉。
\item 当项目在进行了重大修改或局部修正累积较多,而导致项目整体发生全局变化时,主版本号加1。
\end{enumerate}
\subsection{版本信息格式介绍}
\begin{center}
工程名.分支名.主版本号.子版本号.修正版本号.版本类型.编译时间\\
\end{center}

[示例]
\begin{center}
ysf.stm32f1xx.0.0.1.alpha.201609271352
\end{center}

\begin{center}
	\begin{tabular}{|p{5cm}|p{5cm}|}  
		\hline
		格式名称 & 参数介绍 \\ \hline
		工程名 & ysf \\ \hline
		分支名 & stm32f1xx \\ \hline
		主版本号 & 0 \\ \hline
		子版本号 & 0 \\ \hline
		修正版本号 & 1 \\ \hline
		版本类型 & alpha \\ \hline
		编译时间 & 201609271352 \\ \hline
	\end{tabular}
\end{center}