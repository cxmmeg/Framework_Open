\section{注释规范}
规则 <2-4-1>(强制):注释的内容要清楚、明了,含义准确,在代码的功能、意图层次上进行注释。\\

[说明]\\
注释的目的是让读者快速理解代码的意图。\\
注释不是为了名词解释(what),而是说明用途(why)。\\

规则 <2.2-2>(强制):注释应分为两个角度进行,首先是应用角度,主要是告诉使用者如何使用接口(即你提供的函数),其次是实现角度,主要是告诉后期升级、维护的技术人员实现的原理和细节。\\

[说明]\\
每一个产品都可以分为三个层次,产品本身是一个层次,这个层次之下的是你使用的更小的组件,这个层次之上的是你为别人提供的服务。\\
\\
你这个产品的存在的价值就在于把最底层的小部件的使用细节隐藏,同时给最上层的用户提供方便、简洁的使用接口,满足用于需求。从这个角度来看软件的注释,你应该时刻想着你写的注释是给那一层次的人员看的,如果是用户,那么你应该注重描述如何使用,如果是后期维护者,那么你应该注重原理和实现细节。\\
\\
规则 <2.2-3>(强制):修改代码时,应维护代码周边的注释,使其代码和注释一致,不再使用的注释应删除。\\

[说明]\\
注释的目的在于帮助读者快速理解代码使用方法或者实现细节,若注释和代码不一致,会起到相反的作用。建议在修改代码前应该先修改注释。

\subsection{头文件排版}
\textbf{规则 <2-4> (强制):头文件排版内容依次为版权声明、文件信息、包含的头文件、宏定义、类型定义、声明变量、声明函数,且各个种类的内容间空一行。}

\subsection{源文件排版}
\textbf{规则 <2-5> (强制):源文件排版内容依次为版权声明、文件信息、包含的头文件、宏定义、具有外部链接属性的全局变量定义、模块内部使用的static变量、具有内部链接的函数声明、函数实现代码。且各个种类的内容间空三行。}