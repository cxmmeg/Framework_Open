\section{排版规范}
\subsection{缩进}
规则 <2-1> (强制):程序块要采用缩进风格编写,缩进的空格数为4。\\

[说明]\\
函数或过程的开始、结构的定义及循环、判断等语句中的代码都要采用缩进风格,case语句下的情况处理语句也要遵从语句缩进要求(对于由开发工具自动生成的代码可以有不一致)。\\

\subsection{空格空行}
规则 <2-2-1> (强制):进行双目运算、赋值时,操作符之前、之后要加空格;进行非对等操作时,如果是关系密切的立即操作符(如"."),后不应加空格。\\

[示例]\\
\begin{lstlisting}

//! 反例
uint8_t func(void)
{
	struct
	{
		uint8_t a;
		uint8_t b;
	}var;
	
	var.a=0;
	var.b=1;
	
	var.a=var.a+var.b;
	
	return var.a&var.b?1:0;
}

//! 示范代码
uint8_t func(void)
{
	struct
	{
		uint8_t a;
		uint8_t b;
	}var;
	
	var.a = 0;
	var.b = 1;
	
	var.a = var.a + var.b;
	
	return (var.a & var.b) ? (1) : (0);
}
\end{lstlisting}
规则 <2-2-2>(强制):相对独立的程序块之间、变量声明/定义之后必须加空行。\\

[示例]\\
\begin{lstlisting}
//! 反例
uint8_t func(void)
{
	uint8_t a = 0;
	uint8_t b = 1;
	a++;b++;
	return a*b;
}

//! 示范代码
uint8_t func(void)
{
	uint8_t a = 0;
	uint8_t b = 0;
	
	a++;
	b++;
	
	return (a * b);
}
\end{lstlisting}

规则<2-2-3>(强制):较长的语句(大于80字符)要分成多行书写,长表达式要在低优先级操作符处划分新行,操作符放在新行之首,划分出的新行要进行适当的缩进,使排版整齐,语句可读。\\

[示例]\\
\begin{lstlisting}
//! 反例
if (a && b || a && c || c && d || d || e && f)
{
	//! code
}

//! 示范代码
if (   a && b 
	|| a && c 
	|| c && d 
	|| d 
	|| e && f)
{
	//! code
}
\end{lstlisting}

规则 <2-2-4>(强制):若函数或过程中的参数较长,则要进行适当的划分换行。\\

[示例]\\
\begin{lstlisting}
//! 反例
func((uint8_t *)&this_func_param1, (uint16_t)this_func_param2, (uint8_t *)&this_func_param3.address, this_func_param4);

//! 示范代码
func((uint8_t *)&this_func_param1, 
	 (uint16_t)this_func_param2, 
	 (uint8_t *)&this_func_param3.address, 
	 this_func_param4);
\end{lstlisting}

规则 <2-2-5>(强制):不允许把多个短语句写在一行中,即一行只写一条语句。\\

[示例]\\
\begin{lstlisting}
//! 反例
uint8_t a, b, c; char str;

//! 示范代码
uint8_t a, b, c;
char str;
\end{lstlisting}

\subsection{对齐}
规则 <2-3-1>(建议):多行同类操作时操作符尽量保持对齐。\\

[示例]\\
\begin{lstlisting}
int a, aa, aaa;

//! 反例
a = 0;
aa = 0;
aaa = 0;

//! 示范代码
a   = 0;
aa  = 0;
aaa = 0;
\end{lstlisting}

规则 <2.3-2>(强制):if、 for、 do、 while、 case、 switch、 default 等语句自占一行,且 if、 for、 do、 while等语句的执行语句部分无论多少都要加括号{}。\\

[示例]\\
\begin{lstlisting}
//! 反例
if(a)
	a--;
else
	a++;

//! 示范代码
if (a)
{
	a--;
}
else
{
	a++;
}
\end{lstlisting}

规则 <2-3-3>(建议):一行程序以小于80字符为宜,不要写的过长。